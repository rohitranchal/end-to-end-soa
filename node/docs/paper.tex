%\documentclass[journal,10pt,draftclsnofoot,onecolumn]{IEEEtran}
\documentclass[10pt,journal,cspaper,compsoc]{IEEEtran}
%\documentclass[journal,compsoc]{IEEEtran}
\usepackage{graphicx}
\usepackage{amsfonts}
\usepackage{wasysym}
\usepackage{listings}
\usepackage{bm}
\usepackage{booktabs}


\pdfminorversion=4

% correct bad hyphenation here
\hyphenation{op-tical net-works semi-conduc-tor}


\begin{document}
%
% paper title
% can use linebreaks \\ within to get better formatting as desired
\title{Service Interaction Monitoring in SOA Topologies}


% author names and affiliations
% use a multiple column layout for up to three different
% affiliations
\author{\IEEEauthorblockN{Ruchith Fernando, Rohit Ranchal, Bharat Bhargava\\}
\IEEEauthorblockA{Department of Computer Science, Purdue University\\
West Lafayette, IN\\
rfernand@cs.purdue.edu, rranchal@purdue.edu, bb@cs.purdue.edu}}


\IEEEcompsoctitleabstractindextext{%
\begin{abstract}
%\boldmath
This work identifies a set of common service topologies and explores how service trust changes based on various types of service interactions within those topologies. The notions of passive and active service monitors are utilized in minotoring service interactions, which are configured with a set of trust management algorithms. Details of the design and implementation of a RESTful service monitoring framework, and experiments conducted with it are provided. 
\end{abstract}

\begin{keywords}
SOA, REST, Trust
\end{keywords}}

\maketitle

\section{Introduction}
\subsection{SOA Patterns}
\subsection{Passive Monitoring}
\subsection{Trust Propagation}

\section{Design}
\section{Implementation}
\section{Experiments}


\bibliographystyle{unsrt} 
\bibliography{ref}

\end{document}
